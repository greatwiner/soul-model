\documentclass[a4paper,twoside,12pt]{article}
% Alternative Options:
%	Paper Size: a4paper / a5paper / b5paper / letterpaper / legalpaper / executivepaper
% Duplex: oneside / twoside
% Base Font Size: 10pt / 11pt / 12pt

%% Language %%%%%%%%%%%%%%%%%%%%%%%%%%%%%%%%%%%%%%%%%%%%%%%%%
\usepackage[utf8]{inputenc}
\usepackage{ae}

\usepackage{lmodern} %Type1-font for non-english texts and characters

\usepackage[top=2.5cm, bottom=2cm, left=2cm, right=2cm]{geometry}
\usepackage{icomma} % Permet l'utilisation de virgule comme séparateur décimal
\usepackage{url} % Package pour ne pas avoir des problèmes avec des URL's
% Utiliser \url{}

%% Packages for Graphics & Figures %%%%%%%%%%%%%%%%%%%%%%%%%%
%\usepackage[dvips]{graphicx}
%\usepackage{color,psfrag}
\usepackage{graphicx} %%For loading graphic files
%\usepackage{subfig} %%Subfigures inside a figure
%\usepackage{tikz} %%Generate vector graphics from within LaTeX
%\usepackage{tikz-3dplot} %requires 3dplot.sty to be in same directory, or in your LaTeX installation
%\usetikzlibrary{calc} %pour faire les calculs

%\usepackage{epstopdf}

\newcommand{\HRule}{\rule{\linewidth}{0.5mm}} %Dessine la ligne horizontale

%% Please note:
%% Images can be included using \includegraphics{filename}
%% resp. using the dialog in the Insert menu.
%% 
%% The mode "LaTeX => PDF" allows the following formats:
%%   .jpg  .png  .pdf  .mps
%% 
%% The modes "LaTeX => DVI", "LaTeX => PS" und "LaTeX => PS => PDF"
%% allow the following formats:
%%   .eps  .ps  .bmp  .pict  .pntg


%% Math Packages %%%%%%%%%%%%%%%%%%%%%%%%%%%%%%%%%%%%%%%%%%%%
\usepackage{amsmath,amsfonts,amstext,amscd,bezier,amsthm,amssymb}

\newcommand{\vect}[1]{\boldsymbol{#1}}

%\newenvironment{solu}{\noindent {\bf Solution.}\small }{\hfill $\square$ \normalsize \medskip}

%\renewcommand\thesection{\arabic{section}}
%\renewcommand*\thesection{Question \arabic{section}}

\def\MYTITLE{TD processus stochastiques corrections}

\title{\MYTITLE}
\author{\textsc{Do} Quoc Khanh}
\date{\today}

\usepackage{fancyhdr}

\fancyhead[L]{\slshape Quoc Khanh DO}%\slshape\thepage LE,RO
\fancyhead[R]{\slshape TD corrections}%{\slshape \leftmark}
%\fancyhead[LO]{ccc}%{\slshape \rightmark}
%\fancyfoot[LO,LE]{}%\slshape Short Course on Asymptotics
\fancyfoot[C]{\thepage}
%\fancyfoot[RO,RE]{}%\slshape 7/15/2002

\DeclareMathOperator*{\argmax}{arg\,max}
\DeclareMathOperator{\Tr}{Tr}
\DeclareMathOperator{\diag}{diag}
\newcommand{\abs}[1]{{\lvert #1 \rvert}}
\newcommand{\norm}[1]{{\lVert #1 \rVert}}

\begin{document}
\maketitle
\pagestyle{fancy}

\section{Exercice 1}
Deux machines automatiques, indépendantes.
Fiabilité $p$ pour une journée. Lorsqu'elle tomme en panne elle est réparée pendant la nuit et se retrouve donc en état de marche le lendemain. Mais une seule machine peut être réparée à la fois.

$X_n$ le nombre de machines en panne au début de la n-ième journée.

1. Graphe des transitions:

$X_n \in \{0, 1, 2\}$.
\begin{align}
    p_{00} &= p^2 + 2p(1-p) = 2p - p^2 \notag \\
    p_{01} &= (1-p)^2 \notag \\
    p_{10} &= p \notag \\
    p_{11} &= 1-p \notag
\end{align}

2. Matrice de transition:

$ \left( 
\begin{array}{cc}
p(2-p) & (1-p)^2 \\
p & 1-p \\
\end{array} 
\right)$

Explications: pour calculer $p_{00} = p(X_{n+1}=0\vert X_n = 0)$, si au début du jour $n$, aucune machine est en panne, pour que ce soit le même pour le lendemain, il y a deux possibilités:

-Soit il y a aucune machine tombant en panne au cours de la journée $n$, donc cela avec la probabilité $p^2$.

-Soit il y a exactement une machine tombant en panne au cours de la journée $n$, et donc avec la probabilité: $p(1-p)$ (pour la machine $2$ tombant en panne) $+$ $(1-p)p$ (pour la machine $1$ tombant en panne) = $2p(1-p)$.

Donc $p_{00} = p^2 + 2p(1-p) = 2p - p^2 = p(2-p)$.

3. Supposons que les deux machines sont au départ en état de marche donc $X_0 = 0$. Les probabilités au départ sont donc $\pi(0) = (1, 0)$, on a:

\begin{align}
    \pi(1) &= \pi(0).P = (p(2-p), (1-p)^2)\notag \\
    \pi(2) &= \pi(1).P = \pi(0).P^2 = (p(1+2p-3p^2+p^3), (1-p)^2(1+p-p^2)) \notag
\end{align}

\textbf{Distribution limite:} Supposons que $0 < p < 1$, d'après le théorème 2, il existe une distribution limite. On calcule cette distribution $\pi^{*}$.

Cette distribution satisfont les équation de balance (poly page 12). On peut utiliser le graphe de transition en interprétant les probabilités $\pi^{*}_k$ comme des masses associées aux états $k \in S$, et les produits $\pi^{*}_kp_{kj}$ comme des flux de masse entre les états $k$ et $j$. Probabilités stationnaire si et seulement si le flux d'entrée est égal au flux de sortie, donc:
\begin{align}
    \pi^{*}_0.(1-p)^2 = \pi^{*}_1.p \notag
\end{align}

Parce que $\pi^{*}_0 + \pi^{*}_1 = 1$, on a:
\begin{align}
    \dfrac{\pi^{*}_0}{p} = \dfrac{\pi^{*}_1}{(1-p)^2} = \dfrac{1}{p+(1-p)^2} \notag \\
    \Rightarrow \pi^{*} = \left(\dfrac{p}{p+(1-p)^2}, \dfrac{(1-p)^2}{p+(1-p)^2}\right)\notag
\end{align}

\section{Exercice 2:}
Deux éléments indépendants comme à l'exercice 1, sauf que les éléments ne peut pas être réparés. Soit $X_n$ le nombre de machines en panne au début de la n-ième journée.

1. Le graphe des transitions:

$X_n \in \{0, 1, 2\}$ donc $2$ états en total.

$p_{00} = p^2$, $p_{01} = 2p(1-p)$, $p_{02} = (1-p)^2$.

$p_{10} = 0$, $p_{11} = p$, $p_{12} = 1-p$.

$p_{20} = 0$, $p_{21} = 0$ et $p_{22} = 1$.

La matrice des transitions:

$\left(\begin{array}{ccc}
    p^2 & 2p(1-p) & (1-p)^2 \\
    0 & p & 1-p \\
    0 & 0 & 1
\end{array}
\right)$

2. La distribution stationnaire:

Si $p=1$, alors les machines ne tombent jamais, c'est-à-dire que la chaine reste l'état initial. Tous les flux sont zéros, donc tous les état de probabilités sont stationnaires, mais attention il n'y a pas de distribution limite car la chaine est constante.

Si $p<1$, la seule distribution stationnaire est $(0, 0, 1)$. C'est aussi la distribution limite.

3. La chaine est absorbante?

Il y a un état absorbant, mais la chaine est-elle absorbante? Oui si $p < 1$, parce qu'on peut passer de n'importe quel état à cet état absorbant.

4. Prenons $p = 0,9$. Soit $n_i$ temps moyen jusqu'à l'absorption en partant de l'état $i$, $i = 0, 1, 2$. Évidemment $n_2 = 0$. D'après le théorème 7 (page 16), les $n_i$ satisfont les équations suivantes:
\begin{align}
    n_0 &= 1 + p_{00}n_0 + p_{01}n_1 \notag \\
    &= 1 + p^2n_0 + 2p(1-p)n_1 \notag \\
    n_1 &= 1 + p_{10}n_0 + p_{11}n_1 \notag \\
    &= 1 + pn_1 \notag \\
    \Rightarrow n_1 &= \dfrac{1}{1-p} \notag
\end{align}

et on obtient $(1-p^2)n_0 = 1+2p \Rightarrow n_0 = \dfrac{1+2p}{1-p^2}$.

Avec $p=0,9$, on a $n_0 = 14,74$

\section{Exercice 3:}
On ne répare plus une machine tombée en panne que le lendemain.

1. Le graphe des transitions:

$X_n \in \{0, 1, 2\}$.

$p_{00} = p^2$, $p_{01} = 2p(1-p)$, $p_{02} = (1-p)^2$.

$p_{10} = p$, $p_{11} = 1-p$, $p_{12} = 0$.

$p_{20} = 0$, $p_{21} = 1$, $p_{22} = 0$.

2. La matrice des transitions:

$\left(
\begin{array}{ccc}
    p^2 & 2p(1-p) & (1-p)^2 \\
    p & 1-p & 0 \\
    0 & 1 & 0
\end{array}\right)$

3. La distribution stationnaire:

$(1-p^2)\pi_0 = p\pi_1$.

$p\pi_1 = 2p(1-p)\pi_0 + \pi_2$.

$\pi_2 = (1-p)^2\pi_0$.

On a $\pi_0 = \dfrac{1}{(1-p)^2}\pi_2$ et $\pi_1 = \dfrac{1+p}{p(1-p)}\pi_2$. Or $\pi_0 + \pi_1 + \pi_2 = 1$, on a:

$\pi = \dfrac{1}{p^3 - 3p^2 + 2p + 1}\left( p, 1-p^2, p(1-p)^2\right)$.

Est-elle une distribution limite? Avec $p=0$, $(0, 1, 0)$ est une distribution limite. Avec $p = 1$, $(1, 0, 0)$ est une distribution limite. Si $0 < p < 1$, les signes des éléments de $P$ est:

$\left(
\begin{array}{ccc}
    + & + & + \\
    + & + & 0 \\
    0 & + & 0
\end{array}
\right)$

et donc de $P^2$:

$\left(
\begin{array}{ccc}
    + & + & + \\
    + & + & + \\
    + & + & 0
\end{array}
\right)$

et donc $P^3$ contient les éléments strictement positifs. Après le théorème 2, la chaine Markov possède une distribution limite. En plus, cette distribution limite doit satisfaire les équations stationnaires, donc c'est exactement $\pi$ qu'on a trouvée ci-dessus.

\section{Exercice 4:}
$P = \left(
\begin{array}{cc}
    1-a & a \\
    b & 1-b
\end{array}
\right)$. $0 \leq a \leq 1$, $0 \leq b \leq 1$, $0 < a+b < 2$.

1. $P^n = \dfrac{1}{a+b}\left(\begin{array}{cc}
    b & a\\
    b & a
\end{array}\right) + \dfrac{(1-a-b)^2}{a+b}\left(\begin{array}{cc}
    a & -a \\
    -b & b
\end{array}\right)$. Démonstration par récurrence.

2. $\pi(n) = \pi(0).P^n$.

3. Parce que $0 < a+b < 2$.

La distribution limite est: $\dfrac{1}{a+b}.\pi(0).\left(\begin{array}{cc}
    b & a\\
    b & a
\end{array}\right) = \dfrac{1}{a+b}(b, a)$.

4. $\pi$ est aussi une distribution stationnaire. Est-elle unique?

La chaine Markov admet une unique distribution stationnaire si et seulement si elle comprend une seule classe récurrente $\Leftrightarrow$ les deux états sont se communiquent $\Leftrightarrow$ $a > 0$ et $b > 0$.

5. Les comportements déterministes.

\section{Exercice 5:}
$P^4$ satisfait le théorème 2, donc convergente.

La distribution limite est aussi stationnaire. On obtient:

$\left(\dfrac{4}{7}, \dfrac{2}{7}, \dfrac{1}{7}\right)$.

\section{Exercice 6:}
1. Déterminer les classes transitoires et récurrentes? D'abord il faut dessiner le graphe des transitions.

Il y a $4$ états. Les états $3$ et $4$ se communiquent. Les états $3$ et $1$ ne se communiquent pas. Les états $1$ et $2$ ne se communiquent pas. Il y a donc $3$ classes: $\{3, 4\}$, $\{1\}$, et $\{2\}$. Une classe est dite \textsl{transitoire} s'il est possible d'en sortir. Sinon, elle est dite \textsl{récurrente}. Les classes $\{3, 4\}$ et $\{2\}$ sont récurrentes, alors que $\{1\}$ est transitoire.

2. Existe-il une distribution stationnaire? Oui parce que pour une chaine de Markov finie, il existe toujours au moins une distribution stationnaire. Une distribution limite? Non. Parce que d'après le théorème $6$, si la chaine possédait une distribution limite $\pi$, alors $\pi$ serait l'unique distribution de la chaine. La chaine possédait une distribution stationnaire unique, d'après le théorème $5$, la chaine comprendrait une unique classe récurrente, mais il y a ici $2$ classes récurrentes, donc c'est faux.

\section{Exercice 7:}

\section{Exercice 8:}
Ce processus peut être décrit par une chaine Markov de $5$ états: les $3$ états de fabrication, la sortie en bon état (l'état $4$), et la mise à l'écart (l'état $5$). Chaque article passe les $3$ états de fabrication, s'il est défectueux, il est jeté; s'il est en bon état, il passe à la prochaine étape ou quitte la machine en bon état s'il est déjà à la dernière étape de fabrication. S'il est partiellement défectueux, il reste à l'étape en courant encore une fois.

On a: $p_{12} = p_{23} = p_{34} = r$.

$p_{11} = p_{22} = p_{33} = q$.

$p_{15} = p_{25} = p_{35} = p$.

La durée moyenne jusqu'à ce qu'un article quitte la machine $\rightarrow$ temps moyen jusqu'à l'absorption en partant de l'état $1$, donc $n_1$. D'après le théorème $7$, $n_i, i=1,2,3$ satisfont les équations suivantes:
\begin{align}
    n_1 &= 1 + qn_1 + rn_2 \notag \\
    n_2 &= 1 + qn_2 + rn_3 \notag \\
    n_3 &= 1 + qn_3 \notag
\end{align}

En résolvant ces équations, on a:
\begin{align}
    n_3 &= \dfrac{1}{1-q} \notag \\
    n_2 &= \dfrac{1-q+r}{(1-q)^2} \notag \\
    n_1 &= \dfrac{(1-q)^2 + r(1-q+r)}{(1-q)^3}\notag
\end{align}

On calcule $n_1 = 3.7$.

La probabilité qu'en quittant la machine, l'article soit en bon état? C'est $b_{14}$: la probabilité de l'absorption à l'état $4$ si on part de l'état $1$. D'après le théorème $8$, les $b_{14}, b_{24}$ et $b_{34}$ satisfont les équations suivantes:
\begin{align}
    b_{14} &= 0 + qb_{14} + rb_{24} \notag \\
    b_{24} &= 0 + qb_{24} + rb_{34} \notag \\
    b_{34} &= r + qb_{34} \notag
\end{align}

En résolvant ces équations, on obtient alors:
\begin{align}
    b_{34} &= \dfrac{r}{1-q}\notag \\
    b_{24} &= \dfrac{r^2}{(1-q)^2}\notag \\
    b_{14} &= \dfrac{r^3}{(1-q)^3} \notag
\end{align}

On calcule $b_{14} = 0.63$.

\section{Exercice 9:}
La matrice des transitions:
\begin{align}
    \left(\begin{array}{cccc}
        0 & \dfrac{1}{3} & \dfrac{1}{3} & \dfrac{1}{3} \\
        \dfrac{1}{3} & 0 & \dfrac{1}{3} & \dfrac{1}{3} \\
        \dfrac{1}{3} & \dfrac{1}{3} & 0 & \dfrac{1}{3} \\
        \dfrac{1}{3} & \dfrac{1}{3} & \dfrac{1}{3} & 0
    \end{array}\right) \notag
\end{align}

Pour calculer le temps moyen pour atteindre l'état $4$ pour la première fois, en rend cet état absorbant, et puis calculer le temps moyen jusqu'à l'absorption. En rendant l'état $4$ absorbant, on obtient une nouvelle matrice des transitions:
\begin{align}
    \left(\begin{array}{cccc}
        0 & \dfrac{1}{3} & \dfrac{1}{3} & \dfrac{1}{3} \\
        \dfrac{1}{3} & 0 & \dfrac{1}{3} & \dfrac{1}{3} \\
        \dfrac{1}{3} & \dfrac{1}{3} & 0 & \dfrac{1}{3} \\
        0 & 0 & 0 & 1
    \end{array}\right) \notag
\end{align}

D'après le théorème $7$, on obtient les équations suivantes pour $n_1$, $n_2$ et $n_3$:
\begin{align}
    n_1 &= 1+\dfrac{1}{3}n_2 + \dfrac{1}{3}n_3 \notag \\
    n_2 &= 1+\dfrac{1}{3}n_1 + \dfrac{1}{3}n_3\notag \\
    n_3 &= 1 + \dfrac{1}{3}n_1 + \dfrac{1}{3}n_2 \notag
\end{align}

Et on obtient $n_1 = n_2 = n_3 = 3$. Attention qu'il n'y a aucune différence entre les différents points de départ.

Maintenant on transforme l'état $2$ en état absorbant. Pour calculer la probabilité d'atteindre $4$, on transforme aussi $4$ en état absorbant. Donc il y a deux états absorbants. On calcule les probabilités $b_{14}$ et $b_{34}$ de l'absorption en état $4$. D'après le théorème $8$, on a les équations suivantes:
\begin{align}
    b_{14} &= \dfrac{1}{3} + \dfrac{1}{3}b_{34}\notag \\
    b_{34} &= \dfrac{1}{3} + \dfrac{1}{3}b_{14}\notag
\end{align}

On obtient $b_{14} = b_{34} = \dfrac{1}{2}$. Donc $p = \dfrac{1}{2}$.

\section{Exercice 10:}
$X_n$ le nombre de voitures se trouvant dans la station immédiatement après l'instant $n$.

1. $(X_n)$ forme bien une chaine de Markov, parce que:

$X_{n+1} = Y_{n+1}$, si $X_n = 0,1$.

$X_{n+1} = \min(Y_{n+1}+1, 2)$ si $X_n = 2$.

Attention: à tout moment, il y a au plus $3$ voitures dans la station.

2.3. Matrice des transitions:
\begin{align}
    \left(\begin{array}{ccc}
        0,4 & 0,4 & 0,2 \\
        0,4 & 0,4 & 0,2 \\
        0 & 0,4 & 0,6
    \end{array}\right)\notag
\end{align}

4. Considérons le régime stationnaire. Les équations de balance:
\begin{align}
    0,6\pi_0 &= 0,4\pi_1 \notag \\
    0,6\pi_1 &= 0,4\pi_0 + 0,4\pi_2 \notag \\
    0,4\pi_2 &= 0,2\pi_1 + 0,2\pi_0 \notag
\end{align}

On obtient $\pi_1 = \dfrac{3}{2}\pi_0$, $\pi_2 = \dfrac{5}{4}\pi_0$. Et enfin $\pi_0 = \dfrac{4}{15}$, $\pi_1 = \dfrac{6}{15}$ et $\pi_2 = \dfrac{1}{3}$.

La station est inoccupée pendant en moyenne $27\%$ du temps.

5. Le moment où les deux places sont occupées pour la première fois, c'est la première fois que $Y_n = 2$. Parce que si $Y_n = 0$ ou $1$, il ne peut pas y avoir $2$ voitures dans les positions d'attente.

Le première fois que $Y_n=2$ est aussi la première fois que $X_n = 2$ (voir les formules). Donc on doit chercher la durée moyenne jusqu'à ce que $X_n = 2$ pour la première fois. On rend l'état $2$ absorbant, et calculer le temps moyen jusqu'à l'absorption. D'après le théorème $7$, on a:
\begin{align}
    n_0 &= 1 + 0,4n_0 + 0,4n_1 \notag \\
    n_1 &= 1 + 0,4n_0 + 0,4n_1 \notag
\end{align}

On obtient $n_0 = n_1 = 5$.

Pour quoi le temps ne dépend pas de l'état initial? C'est parce que c'est le premier moment où $Y_n = 2$, cette variable aléatoire ne dépend pas de l'état de $X_n$. On peut le calculer directement. $(Y_n)$ est aussi une chaine de Markov avec la matrice des transitions:
\begin{align}
    \left(\begin{array}{ccc}
        0,4 & 0,4 & 0,2 \\
        0,4 & 0,4 & 0,2 \\
        0,4 & 0,4 & 0,2
    \end{array}\right)
\end{align}

En appliquant les mêmes démarches pour $(Y_n)$, on obtient le même résultat.

\end{document}